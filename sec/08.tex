\section{연속확률변수}
\defn{어떤 구간에 속하는 모든 실수 값을 취할 수 있는 확률변수를 \textbf{연속확률변수}(continuous random variable)라 한다. 연속확률변수 $X$가 구간 $[\alpha, \beta]$에 속하는 모든 실수 값을 취하고, $$\pr(a\leq X\leq b) = \int_a^b f(x)dx \quad (\alpha \leq a\leq x\leq b\leq \beta)$$ 와 같이 나타낼 수 있을 때, 함수 $f(x)$를 $X$의 \textbf{확률밀도함수}(probability density function)라 하며, 확률밀도함수는 다음 조건을 만족해야 한다.
\begin{enumerate}
	\item[(1)] $f(x) \geq 0$
	\item[(2)] $\ds \int_\alpha^\beta f(x)dx= 1$
	\item[(3)] $\alpha \leq a\leq x\leq b\leq \beta$ 일 때, $$
	\pr(a\leq X\leq b) = \pr(X\leq b) - \pr(X < a) = \int_\alpha^b f(x)dx \, - \,\int_\alpha^a f(x)dx = \int_a^b f(x)dx$$
\end{enumerate}
}

\defn{연속확률변수 $X$가 구간 $[\alpha, \beta]$에 속하는 모든 실수 값을 취할 때,
\begin{itemize}
	\item \textbf{평균}(mean), \textbf{기댓값}(expectation): $\mu = \ex(X) = \ds \int_\alpha^\beta xf(x)dx$\vspace{-3mm}
	\item \textbf{분산}(variance): $\var(X) = \ex((X-\mu)^2) = \ds \int_\alpha^\beta (x-\mu)^2f(x)dx $
	\item \textbf{표준편차}(standard deviation): $\sigma(X) = \sqrt{\var(X)}$
\end{itemize}
}

\prob{연속확률변수 $X$의 확률밀도함수가 $f(x) = ax \: (0\leq x\leq 2)$ 일 때, 상수 $a$의 값과 $\pr(0.5\leq X\leq 1)$의 값을 구하고, $\ex(X)$와 $\var(X)$의 값을 구하여라.}

\defn{연속확률변수 $X$가 모든 실수 값을 취하고, 확률밀도함수 $f(x)$가 다음과 같이 주어질 때, $$f(x) = \frac{1}{\sqrt{2\pi} \sigma} e^{-\frac{(x-\mu)^2}{2\sigma^2}} \qquad (-\infty < x<\infty)$$ $X$의 확률분포를 \textbf{정규분포}(normal distribution)라 하고, 평균이 $\mu$, 분산이 $\sigma^2$인 정규분포를 기호로 $\mc{N}(\mu, \sigma^2)$와 같이 나타낸다.}

\thm{서로 독립인 확률변수 $X_i \sim \mc{N}(\mu_i, \sigma_i^2)$, 상수 $c_i$ $(i = 1,\dots, k)$ 에 대하여 $$X = \sum_{i=1}^k c_iX_i \sim \mc{N}\left(\sum_{i=1}^k c_i\mu_i,\: \sum_{i=1}^k c_i^2 \sigma_i^2\right)$$ 이 성립한다.}
\begin{center}
\begin{tikzpicture}

\begin{axis}[
title={표준정규분포 $\mc{N}(0, 1)$},
mark=none, domain=0:10, samples=100,smooth,
axis lines*=none, axis y line=none,
height=6cm, width=10cm,
xticklabels={-4, -3, -2, -1, 0, 1, 2, 3}, ytick=\empty,
enlargelimits=false, clip=false, axis on top,
grid = major]
\addplot [fill=cyan!20, draw=none, domain=-3:3] {gauss(0,1)} \closedcycle;
\addplot [fill=orange!20, draw=none, domain=-3:-2] {gauss(0,1)} \closedcycle;
\addplot [fill=orange!20, draw=none, domain=2:3] {gauss(0,1)} \closedcycle;
\addplot [fill=blue!20, draw=none, domain=-2:-1] {gauss(0,1)} \closedcycle;
\addplot [fill=blue!20, draw=none, domain=1:2] {gauss(0,1)} \closedcycle;
\end{axis}
\end{tikzpicture}	
\end{center}


\defn{$\mc{N}(0, 1)$ 을 \textbf{표준정규분포}(standard normal distribution)라 하고, 확률밀도함수 $f(z)$는 다음과 같다. $$f(z) = \frac{1}{\sqrt{2\pi}}e^{-\frac{z^2}{2}} \qquad (-\infty < z <\infty)$$}\\\\

\thm{(\textbf{정규분포의 표준화}) $X\sim \mc{N}(\mu, \sigma^2)$ 일 때, $$Z = \frac{X-\mu}{\sigma} \sim \mc{N}(0, 1)$$가 성립하고, 이를 \textbf{정규분포의 표준화}(standardization)라 한다. 이렇게 표준화된 값을 \textbf{$z$-점수}($z$-score)라 하고, 다음이 성립한다. $$\pr(a\leq X\leq b) = \pr\left(\frac{a-\mu}{\sigma} \leq Z\leq \frac{b-\mu}{\sigma}\right)$$}\\

\thm{표준정규분포 $\mc{N}(0, 1)$ 의 확률밀도함수 $f(z)$는 다음과 같은 성질을 갖는다.
\begin{itemize}
	\item 곡선과 $x$축 사이의 넓이는 $1$이다.
	\item 직선 $x=0$ 에 대하여 대칭이다.
\end{itemize}이로부터 다음이 성립함을 알 수 있다. 임의의 실수 $a, b$ 에 대하여 ($a\leq b$)
\begin{enumerate}
	\item[(1)] $\pr(Z\geq a) = \pr(Z \leq -a)$.
	\item[(2)] $\pr(Z\geq a) = 1-\pr(Z<a)$
	\item[(3)] $\pr(a\leq Z\leq b) = \pr(Z\leq b) - \pr(Z < a)$
\end{enumerate}
}

\prob{표준정규분포표가 주어져 있다. $X\sim \mc{N}(27, 4^2)$ 일 때, 다음을 구하여라.
\begin{center}
	\begin{tabular}{c|c}
		$z$ & $\pr(0\leq Z\leq z)$ \\ \hline
		0.5 & 0.1915 \\
		1.0 & 0.3413 \\
		1.5 & 0.4332 \\
		2.0 & 0.4772
	\end{tabular}
\end{center}
\begin{enumerate}
	\item[(1)] $\pr(X\leq 21)$
	\item[(2)] $\pr(29\leq X\leq 35)$
	\item[(3)] $\pr(X\geq 25)$
\end{enumerate}
}


\defn{$\mc{N}(0, 1)$ 의 $100(1-\alpha)$ 백분위수를 $z_\alpha$ 로 나타낸다. 즉 $\pr(Z\geq z_\alpha) = \alpha$ 이다.\footnote{\textbf{상방백분위수}라고도 한다.}}

\prob{$\pr(Z<1.96) = 0.975$, $\pr(Z < 2.58) = 0.995$ 일 때, $z_{0.025}, z_{0.005}$ 의 값을 구하여라.}\\

\thm{(68.26-95.44-99.74 Rule) $X\sim \mc{N}(\mu, \sigma^2)$ 일 때,
\begin{itemize}
	\item 68.26\% 의 관측값들이 $\left[\mu - \sigma , \mu +\sigma\right]$ 에 있다. $\pr\left(\left|X-\mu\right|<\sigma\right) = 0.6826$.
	\item 95.44\% 의 관측값들이 $\left[\mu - 2\sigma , \mu +2\sigma\right]$ 에 있다. $\pr\left(\left|X-\mu\right|<2\sigma\right) = 0.9544$.
	\item 99.74\% 의 관측값들이 $\left[\mu - 3\sigma , \mu + 3\sigma\right]$ 에 있다. $\pr\left(\left|X-\mu\right|<3\sigma\right) = 0.9974$.
\end{itemize}
}

\prob{전세계 사람들의 IQ는 평균이 100, 표준편차가 16인 정규분포를 따른다고 한다. IQ가 116, 132, 148인 사람은 각각 IQ 상위 몇 \% 인지 구하여라.}\\\\

\prob{대수능 모의평가에서 어느 고등학교 3학년 학생 500명의 수학 성적이 평균 70점, 표준편차 20점인 정규분포를 따른다고 한다. 300등을 한 학생의 점수를 구하여라.
\begin{center}
	\begin{tabular}{c|c}
		$z$ & $\pr(0\leq Z\leq z)$ \\ \hline
		0.25 & 0.1 \\
		0.52 & 0.2 \\
		1.28 & 0.4
	\end{tabular}
\end{center}
}\\

\prob{다음은 9등급제의 산출 방식 중 일부이다.
\begin{center}
	\begin{tabular}{c|c|c}
		등급 & $z$ & $\pr(Z\geq z)$ \\ \hline
		1 & 1.75 & 0.04 \\ 2 & 1.25 & 0.11 \\ 3 & 0.75 & 0.23 \\ 4 & 0.25 & 0.40
	\end{tabular}
\end{center}
어떤 시험의 평균이 50점이고 표준편차가 18점일 때, 각 등급컷 점수를 구하여라.
}\\\\\\\\

\thm{(\textbf{드 무아브르-라플라스의 정리}) $X\sim \mr{B}(n, p)$ 일 때, $n$이 충분히 크면\footnote{일반적으로, $np\geq 5, n(1-p)\geq 5$ 이면 근사한다.} $X$는 근사적으로 $\mc{N}(np, np(1-p))$ 를 따른다.}

\thm{(\textbf{연속성 수정} - continuity correction) 연속확률분포를 이용하여 이산확률분포의 확률을 근사시킬 때, 근사의 정밀도를 높이는데 사용한다. $X\sim \mr{B}(n, p)$ 일 때, $$\pr(a\leq X\leq b) \approx \pr\left(\frac{a-np \mathbf{-0.5}}{\sqrt{np(1-p)}} \leq Z \leq\frac{a-np \mathbf{+0.5}}{\sqrt{np(1-p)}} \right)$$}

\prob{현재 20살인 사람이 45년 후 살아있을 확률이 0.8 이라고 한다. 20살인 사람 500명을 임의로 추출했을 때, 다음 값을 식으로 표현하여라.
\begin{enumerate}
	\item[(1)] 45년 후, 정확히 390명이 살아있을 확률
	\item[(2)] 45년 후, 375명 이상 425명 이하의 사람들이 살아있을 확률
\end{enumerate}
}
\\\\\\\\\\\\
\pagebreak
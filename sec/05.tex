\section{조건부확률}
\defn{확률이 $0$이 아닌 두 사건 $A, B$에 대하여 사건 $A$가 일어났을 때, 사건 $B$가 일어날 확률을 사건 $A$가 일어났을 때의 사건 $B$의 \textbf{조건부확률}(conditional probability)이라 하고, 기호로 $\pr(B\,|\,A)$ 와 같이 나타낸다. 이는 다음과 같이 계산한다.$$\pr(B\,|\,A) = \frac{\pr(A\cap B)}{\pr(A)} \quad \big(\pr(A)>0\big)$$}

\thm{\textbf{(확률의 곱셈정리)} 공사건이 아닌 두 사건 $A, B$에 대하여 다음이 성립한다.$$\pr(A\cap B) = \pr(B\,|\,A)\pr(A) = \pr(A\,|\,B)\pr(B)$$}

\defn{두 사건 $A, B$에 대하여 사건 $A$가 일어났을 때의 사건 $B$의 조건부확률이 사건 $B$가 일어날 확률과 같을 때, 즉 $$\pr(B\,|\,A) = \pr(B\,|\,A^C) = \pr(B)$$ 이면, 두 사건 $A, B$는 서로 \textbf{독립}(independent)이라 하고, 기호로 $A\indep B$ 와 같이 나타낸다. 두 사건이 독립이 아닐 때는 \textbf{종속}이라 한다.}

\thm{공사건이 아닌 두 사건 $A, B$에 대하여 다음 조건은 서로 동치이다.
\begin{enumerate}
	\item[(1)] $A, B$가 서로 독립이다. \vspace{-3mm}
	\item[(2)] $\pr(A\cap B) = \pr(A)\pr(B)$
\end{enumerate}}

\thm{공사건이 아닌 두 사건 $A, B$에 대하여 다음이 성립한다.
\begin{center}
	[$A, B$가 독립] $\iff$ [$A^C, B$가 독립] $\iff$ [$A, B^C$가 독립] $\iff$ [$A^C, B^C$가 독립]
\end{center}}

\defn{공사건이 아닌 사건 $A_1, \dots, A_n$에 대하여, $$\pr(A_i\cap A_j) = \pr(A_i)\pr(A_j) \quad \text{for all }1\leq i\neq j\leq n$$
	이 성립하면 사건 $A_1, \dots, A_n$이 \textbf{쌍마다 독립}(pairwise independent)이라고 한다. }

\defn{공사건이 아닌 사건 $A_1, \dots, A_n$에 대하여, $$\pr\left(\bigcap_{i=1}^n A_i\right) = \prod_{i=1}^n \pr(A_i)$$
	이 성립하면 사건 $A_1, \dots, A_n$이 \textbf{상호 독립}(mutually independent)이라고 한다. }

\defn{동일한 시행을 반복할 때, 각 시행에서 일어나는 사건이 서로 독립이면 이러한 시행을 \textbf{독립시행}이라고 한다.}

\thm{\textbf{(독립시행의 확률)} $1$회의 시행에서 사건 $A$가 일어날 확률을 $p$ 라 할 때, 이 시행을 $n$회 반복하는 독립시행에서 사건 $A$가 $r$번 일어날 확률은 다음과 같다. $${n\choose r} p^r(1-p)^r \quad \big(r=0, 1, \dots, n\big)$$}

\thm{(\textbf{전확률공식} - Law of Total Probability) 표본공간 $S$의 분할인 사건 $A_1, \dots, A_n$에 대하여 다음이 성립한다.
$$\pr(B) = \sum_{i=1}^n \pr(B\, |\, A_i)\,\pr(A_i)$$
\textbf{증명.} $\ds \pr(B) = \pr\left( \bigcup_{i=1}^n\left(B\cap A_i\right)\right) = \sum_{i=1}^n \pr(B\cap A_i) = \sum_{i=1}^n\pr(B\, |\, A_i)\,\pr(A_i) $ \qed
}

\thm{(\textbf{베이즈 정리} - Bayes' Theorem) 사건 $A_1, \dots, A_n$이 표본공간 $S$의 분할이고 $\pr(B)>0$ 이면 다음이 성립한다.
$$\pr(A_k \, |\, B) = \frac{\pr(B\,|\,A_k)\,\pr(A_k)}{\ds\sum_{i=1}^n \pr(B\, |\, A_i)\,\pr(A_i)}$$
\textbf{증명.} 조건부확률의 정의와 전확률공식으로부터 자명. \qed
}


\pagebreak

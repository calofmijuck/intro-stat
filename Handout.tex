\documentclass[12pt]{article}
\usepackage{graphicx}

\usepackage{kotex}
\usepackage{amsmath}
\usepackage{amsfonts}
\usepackage{amssymb}
\usepackage{mathtools}
\usepackage{geometry}
\usepackage{amsthm}
\geometry{
	top = 30mm,
	left = 25mm,
	right = 25mm,
	bottom = 30mm
}
\geometry{a4paper}

\renewcommand{\baselinestretch}{1.4}


\theoremstyle{definition}
\newtheorem{theorem}{\sffamily 정리}[section]
\theoremstyle{definition}
\newtheorem{definition}[theorem]{\sffamily 정의}  
\newtheorem{exmp}[theorem]{\sffamily 예제}

\newcommand{\defn}[1]{\begin{definition}#1\end{definition}~}
\newcommand{\thm}[1]{\begin{theorem}{#1}\end{theorem}~}

\usepackage{titlesec}

\title{\textbf{AP Statistics}}
\author{Sungchan Yi}
\date{January 2019}

\titleformat*{\section}{\large\bfseries}

\newcommand{\p}[2]{_{#1}\text{P}_{#2}}
\newcommand{\ds}{\displaystyle}
\newcommand{\pr}{\text{P}}
\newcommand{\indep}{\mathrel{\!\perp \!\!\!\perp\!}}

\begin{document}
\maketitle

\tableofcontents
\pagebreak

\section{자료의 생성}
\textbf{통계학}(statistics)이란, 주어진 문제에 대하여 합리적인 답을 줄 수 있도록 숫자로 표시되는 정보를 \textbf{수집}하고 정리하며, 이를 해석하고 \textbf{신뢰성 있는 결론}을 이끌어 내는 방법을 연구하는 과학의 한 분야이다.\\\\
그러면 두 가지 질문이 생긴다.
\begin{enumerate}
	\item \textbf{수집}: 어떻게 수집해야 전체를 잘 대표할 수 있는가?
	\item \textbf{신뢰성 있는 결론}: 어떻게 신뢰성을 측정하여 결론을 내릴 것인가?
\end{enumerate}
\vspace{3mm}
\defn{~
\begin{itemize}
	\item \textbf{추출단위}(sampling unit): 전체를 구성하는 각 개체
	\item \textbf{특성값}(characteristic): 각 추출단위의 특성을 나타내는 값
	\item \textbf{모집단}(population): 관심의 대상이 되는 모든 추출단위의 특성값을 모아 놓은 것\\ 추출단위의 개수가 유한하면 \textbf{유한모집단}, 무한하면 \textbf{무한모집단}이라 한다.
	\item \textbf{표본}(sample): 실제로 관측한 추출단위의 특성값의 모임
\end{itemize}
}

\defn{(자료의 종류)
	\begin{enumerate}
		\item \textbf{범주형 자료}(categorical data), \textbf{질적 자료}(qualitative data)는 관측 결과가 몇 개의 범주 또는 항목의 형태로 나타나는 자료이다.
		\begin{itemize}
			\item \textbf{명목자료}(nominal data): 순위의 개념이 없다. 예) 혈액형, 성별
			\item \textbf{순서자료}(ordinal data): 순위의 개념을 갖는다. 예) A $\sim$ F 학점, 9등급제
		\end{itemize}
		\item \textbf{수치형 자료}(numerical data), \textbf{양적 자료}(quantitative data)는 자료 자체가 숫자로 표현되며 숫자 자체가 자료의 속성을 반영한다.
		\begin{itemize}
			\item \textbf{이산형 자료}(discrete data) 예) 교통사고 건수
			\item \textbf{연속형 자료}(continuous data) 예) 키, 몸무게
		\end{itemize}
	\end{enumerate}}

\defn{(통계학의 분류)
\begin{itemize}
	\item \textbf{기술통계학}(descriptive statistics)은 표나 그림 또는 대표값 등을 통하여 수집된 자료의 특성을 쉽게 파악할 수 있도록 자료를 정리$\cdot$요약 하는 방법을 다루는 분야이다.
	\item \textbf{추측통계학}(inferential statistics)은 표본에 내포된 정보를 분석하여 모집단의 여러가지 특성에 대하여 과학적으로 추론하는 방법을 다루는 분야이다.	
\end{itemize}
}

\defn{$N$개의 추출단위로 구성된 유한모집단에서 $n$개의 추출단위를 비복원추출할 때, $_N \mathrm{C}_n$개의 모든 가능한 표본들이 동일한 확률로 추출되는 방법을 \textbf{단순랜덤추출법}(simple random sampling)이라 하고, 이 방법을 위해서는 난수표(random number table)나 난수생성기(random number generator) 등을 이용한다. 그리고 단순랜덤추출로 얻은 표본을 \textbf{단순랜덤표본}(simple random sample)이라 한다. 
}

\defn{(통계적 실험)
\begin{itemize}
	\item 실험이 행해지는 개체를 \textbf{실험단위}(experimental unit/subject)라 하고, 각각의 실험단위에 특정한 실험환경 또는 실험조건을 가하는 것을 \textbf{처리}(treatment)라 한다.
	\item 처리를 받는 집단을 \textbf{처리집단}(treatment group), 처리를 받지 않은 집단을 \textbf{대조집단}(control group)이라 한다.
	\item 실험환경이나 실험조건을 나타내는 변수를 \textbf{인자}(factor)라 하고, 인자가 취하는 값을 그 인자의 \textbf{수준}(level)이라 한다. 
	\item 인자에 대한 반응을 나타내는 변수를 \textbf{반응변수}(response variable)라 한다.
	\item 실험단위가 처리집단이나 대조집단에 들어갈 기회를 동등하게 부여하는 방법을 \textbf{랜덤화}(randomization)라 한다.
	\item 랜덤화에 의해 모든 실험단위를 각 처리에 배정하는 실험계획을 \textbf{완전 랜덤화 계획}(completely randomized design)이라 한다.
	\item 실험 이전에 동일 처리에 대한 반응이 유사할 것으로 예상되는 실험단위들끼리 모은 것을 \textbf{블록}(block)이라 하고, 랜덤화에 의해 모든 블록을 각 처리에 배정하는 실험계획을 \textbf{블록화}(randomized block design)라 한다.
\end{itemize}
}

\defn{\textbf{(통계적 실험계획의 원칙)}
\begin{enumerate}
	\item \textbf{대조}(control): 관심 인자 이외의 다른 외부 인자의 효과를 극소화하고, 처리에 대한 대조집단을 통해 비교 실험을 한다.
	\item \textbf{랜덤화}(randomization): 완전랜덤화계획
	\item \textbf{반복 시행}(replication): 처리효과의 탐지를 용이하게 하기 위해 반복 시행한다. 
\end{enumerate}
}


\pagebreak

\section{대표값과 산포도}

\pagebreak
\section{순열과 조합}
\defn{$0!=1$, $n! = \displaystyle \prod_{i=1}^n i  = n\cdot(n-1)\cdot\cdots\cdot2\cdot 1 \: (n\geq 1)$ 로 정의하고, $!$ 는 \textbf{팩토리얼}(factorial)이라 읽는다.}
\defn{서로 다른 $n$개의 원소에서 서로 다른 $r$개를 택하여 일렬로 배열하는 것을 $n$개에서 $r$개를 택하는 \textbf{순열}(permutation)이라 하고, 기호로 $\p{n}{r}$ 와 같이 나타낸다.}

\thm{$\p{n}{r} = n(n-1)\cdots(n-r+1) = \dfrac{n!}{(n-r)!}$ \quad (단, $0\leq r\leq n$)}

\defn{서로 다른 $n$개의 원소에서 순서를 생각하지 않고 $r$개를 택하는 것을 $n$개에서 $r$개를 택하는 \textbf{조합}(combination)이라 하고, 기호로 $_{n}\text{C}_{r}$ 또는 $\displaystyle {n \choose r}$ 과 같이 나타낸다.}

\thm{$\displaystyle {n \choose r} = \dfrac{\p{n}{r}}{r!} = \dfrac{n!}{r!(n-r)!}$ \quad (단, $0\leq r\leq n$)}

\thm{(조합의 성질) 
\begin{enumerate}
	\item[(1)] $\displaystyle {n \choose r} = {n \choose n - r}$ \quad (단, $0\leq r\leq n$) (대칭성) 
	\item[(2)] $\ds {n\choose r} = {n-1\choose r} + {n-1\choose r-1} $ \quad (단, $1\leq r\leq n-1$) \textbf{(파스칼 법칙)} 
\end{enumerate}}

\defn{서로 다른 $n$개의 원소에서 중복을 허락하여 $r$개를 택하는 순열을 $n$개에서 $r$개를 택하는 \textbf{중복순열}이라 하고, 기호로 $_n \Pi_r$ 과 같이 나타낸다.}

\defn{서로 다른 $n$개의 원소에서 중복을 허락하여 $r$개를 택하는 조합을 $n$개에서 $r$개를 택하는 \textbf{중복조합}이라 하고, 기호로 $_n \text{H}_r$ 과 같이 나타낸다.}

\thm{$\ds _n \Pi_r = n^r, \quad _n \text{H}_r = {n+r-1 \choose r}$.}

\thm{$n\in\mathbb{N}$ 에 대하여, $$(x+y)^n = \sum_{r=0}^n {n\choose r} x^{n-r}y^{r} = {n \choose 0}x^ny^0 + {n\choose 1}x^{n-1}y^1+\cdots + {n\choose n}x^0y^n$$ 이다. 이를 $ (a+b)^n $에 대한 \textbf{이항정리}(binomial theorem)라 하고, $\ds{n\choose r}x^{n-r}y^r$을 전개식의 \textbf{일반항}, 전개식의 각 항의 계수 $\ds {n\choose r} $들을 \textbf{이항계수}라 한다.}

\thm{(이항계수의 성질)
\begin{enumerate}
	\item[(1)] $\ds (1+x)^n = \sum_{r=0}^n {n\choose r}x^r={n \choose 0} + {n\choose 1}x + \cdots + {n\choose n}x^n$ \quad (for all $x\in \mathbb{C}$)
	\item[(2)] $\ds \sum_{r=0}^n {n\choose r} = {n \choose 0} + {n \choose 1}+\cdots + {n \choose n} = 2^n$
	\item[(3)] $\ds \sum_{r=0}^n (-1)^r {n\choose r} = {n \choose 0} - {n \choose 1} + {n \choose 2} - {n \choose 3} +\cdots + (-1)^n{n \choose n} = 0$
	\item[(4)] $\ds \sum_{r=0}^nr{n\choose r} ={n \choose 1} + 2\cdot {n \choose 2} + \cdots + n\cdot {n \choose n} = n\cdot 2^{n-1}$
	\item[(5)] $\ds \sum_{r=0}^nr^2{n\choose r} = 2^2\cdot {n \choose 2} +3^2\cdot {n \choose 3} + \cdots + n^2\cdot {n \choose n} = n(n+1)\cdot 2^{n-2}$
	\item[(6)] $\ds \sum_{r=0}^n \frac{1}{r+1}{n\choose r}=\frac{1}{1}{n\choose 0} + \frac{1}{2}{n\choose 1} + \cdots + \frac{1}{n+1}{n\choose n} = \frac{1}{n+1}\left(2^{n+1}-1\right)$
\end{enumerate}}

\thm{$n\in\mathbb{N}$ 에 대하여, $$(x_1+x_2+\cdots+x_m)^n = \sum_{r_1+r_2+\cdots+r_m=n} {n\choose r_1, r_2, \dots, r_m}x_1^{r_1}x_2^{r_2}\cdots x_m^{r_m} $$ 이고, 이를 $(x_1+x_2+\cdots+x_m)^n$에 대한 \textbf{다항정리}(multinomial theorem)라 한다. 이 때 $\ds{n\choose r_1, r_2, \dots, r_m}$를 \textbf{다항계수}라 하고, 다음과 같이 정의한다. $${n\choose r_1, r_2, \dots, r_m} = \frac{n!}{r_1!\cdot r_2!\cdot \cdots \cdot r_m!}$$}

\defn{서로 다른 $n$개의 원소를 원형으로 배열하는 순열을 \textbf{원순열}이라 하고, 그 경우의 수는 $(n-1)!$ 이다.}

\thm{(원순열의 일반공식) $n$개 중에서 서로 같은 것이 $p_1, p_2, \dots, p_k$개씩 있을 때, 이 $n\,(=p_1+\cdots+p_k)$개를 원형으로 배열하는 방법(원순열)의 수는 다음과 같다.
$$\frac{1}{n}\sum_{d\,|\,g} \left\{\phi(d) {\frac{n}{d}\choose \frac{p_1}{n},\frac{p_2}{n},\dots,\frac{p_k}{n}}  \right\}$$
단, $g = \gcd(p_1, \dots, p_k)$, $d>0$ 이고 $\phi(d)$는 $d$ 이하의 자연수 중에서 $d$ 와 서로소인 자연수의 개수로 정의된다. 
}
\pagebreak
\section{확률의 뜻과 활용}
확률은 모집단에서 표본을 추출할 때, 특정 성질을 만족하는 표본이 관측될 가능성에 대한 측도로, 표본을 바탕으로 \textbf{모집단에 대한 결론을 이끌어낼 때 논리적 근거}가 된다.
\defn{같은 조건 아래에서 반복할 수 있고, 그 결과가 우연에 의하여 결정되는 실험이나 관찰을 \textbf{시행}이라고 한다. 어떤 시행에서 일어날 수 있는 모든 가능한 결과 전체의 집합을 \textbf{표본공간}(sample space $S$)이라 하고, 표본공간의 부분집합을 \textbf{사건}(event)이라고 한다.}

\defn{표본공간의 부분집합 중에서 원소의 개수가 한 개인 집합을 \textbf{근원사건}이라 하고, 반드시 일어나는 사건은 \textbf{전사건}, 절대로 일어나지 않는 사건은 \textbf{공사건}($\varnothing$)이라 한다.}


\defn{두 사건 $A, B$에 대하여, $A$ 또는 $B$가 일어나는 사건을 $A$와 $B$의 \textbf{합사건}이라 하고, $A\cup B$ 로 나타낸다. 그리고 $A$와 $B$가 동시에 일어나는 사건을 $A$와 $B$의 \textbf{곱사건}이라 하고, $A\cap B$ 로 나타낸다.}

\defn{표본공간 $S$의 부분집합인 두 사건 $A, B$에 대하여 $A\cap B = \varnothing$ 이면 $A$와 $B$는 서로 \textbf{배반사건}(disjoint)이라 한다. 또, 사건 $A$가 일어나지 않는 사건을 사건 $A$의 \textbf{여사건}이라 하고, $A^C$로 나타낸다.}

\defn{표본공간 $S$의 공사건이 아닌 사건 $A_1, \dots, A_n$이 다음 조건을 만족하면,
\begin{enumerate}
	\item[(1)] $\ds \bigcup_{i=1}^n A_i = S$
	\item[(2)] $A_i\cap A_j = \varnothing\:$  $\big(\text{for all }1\leq i \neq j \leq n\big)$ \quad (pairwise disjoint)
\end{enumerate} 사건 $A_1, \dots, A_n$을 $S$의 \textbf{분할}(partition)이라 한다.
}

\defn{어떤 시행에서 사건 $A$가 일어날 가능성을 수로 나타낸 것을 사건 $A$가 일어날 \textbf{확률}이라 하고, 기호로 $\pr(A)$와 같이 나타낸다.}

\defn{(수학적 확률) 어떤 시행의 표본공간 $S$가 $m$개의 근원사건으로 이루어져 있고, \textbf{각 근원사건이 일어날 가능성이 모두 같은 정도로 기대될 때}, 사건 $A$가 $r$개의 근원사건으로 이루어져 있으면 사건 $A$가 일어날 확률은 다음과 같다.$$\pr(A) = \frac{\text{(사건 }A\text{가 일어나는 경우의 수)}}{\text{(모든 경우의 수)}} =\frac{n(A)}{n(S)} = \frac{r}{m}$$}

\defn{(통계적 확률) 같은 시행을 $n$번 반복하여 사건 $A$가 일어난 횟수를 $r_n$이라고 하자. 이 때, 시행 횟수 $n$이 한없이 커짐에 따라 그 상대도수 $r_n/n$ 은 $\pr(A)$에 가까워진다. $$\pr(A)=\lim_{n\rightarrow \infty} \frac{r_n}{n}$$} 

\defn{(기하학적 확률) 연속적인 변량을 크기로 갖는 표본공간의 영역 $S$ 안에서 각각의 점을 잡을 가능성이 같은 정도로 기대될 때, 영역 $S$에 포함되어 있는 영역 $A$에 대하여 영역 $S$에서 임의로 잡은 점이 영역 $A$에 속할 확률은 다음과 같다. $$\pr(A) = \frac{\text{(영역 }A\text{의 크기)}}{\text{(영역 }S\text{의 크기)}}$$ }

\defn{(확률의 공리 - Axioms of Probability) 표본공간 $S$와 사건 $A$에 대하여,
\begin{enumerate}
	\item[(1)] $0\leq \pr(A)\leq 1$
	\item[(2)] $\pr(S) = 1$ \vspace{-1.5mm}
	\item[(3)] 서로 배반인 사건열 $A_1, A_2, \dots$ 에 대해 $\ds \pr\left(\bigcup_{i=1}^\infty A_i\right) = \sum_{i=1}^\infty \pr(A_i)$
\end{enumerate}
}

\thm{\textbf{(확률의 기본 성질)} 사건 $A, B$에 대하여 다음이 성립한다.
\begin{enumerate}
	\item[(1)] $\pr(\varnothing) = 0$
	\item[(2)] $\pr(A\cup B) = \pr(A) + \pr(B) - \pr(A\cap B)$ \quad \textbf{(확률의 덧셈정리)}
	\item[(3)] $\pr(A^C) = 1-\pr(A)$ \quad (여사건의 확률)

\end{enumerate}
}

\thm{사건 $A, B, C$에 대하여 다음이 성립한다. $$\pr(A\cup B\cup C) = \pr(A) + \pr(B) +\pr(C) - \pr(A\cap B)-\pr(B\cap C) - \pr(C\cap A) + \pr(A\cap B\cap C)$$}
\thm{\textbf{(포함 배제 원리)} 사건 $A_1, \dots, A_n$에 대하여 다음이 성립한다.
$$\pr\left(\bigcup_{i=1}^n A_i\right) = \sum_{k=1}^n (-1)^{k+1} \left(\sum_{1\leq i_1<\cdots< i_k\leq n} \pr\left(A_{i_1}\cap \cdots \cap A_{i_k}\right)\right)$$
}

\pagebreak

\section{조건부확률}
\defn{확률이 $0$이 아닌 두 사건 $A, B$에 대하여 사건 $A$가 일어났을 때, 사건 $B$가 일어날 확률을 사건 $A$가 일어났을 때의 사건 $B$의 \textbf{조건부확률}(conditional probability)이라 하고, 기호로 $\pr(B\,|\,A)$ 와 같이 나타낸다. 이는 다음과 같이 계산한다.$$\pr(B\,|\,A) = \frac{\pr(A\cap B)}{\pr(A)} \quad \big(\pr(A)>0\big)$$}

\prob{주사위 한 개를 던지는 시행에서 소수의 눈이 나오는 사건을 $A$, 홀수의 눈이 나오는 사건을 $B$라 할 때, 다음을 구하여라.
\begin{enumerate}
	\item[(1)] $\pr(A\cap B)$
	\item[(2)] $\cndpr{B}{A}$
	\item[(3)] $\cndpr{A^C}{B}$
\end{enumerate}
} 
\thm{\textbf{(확률의 곱셈정리)} 공사건이 아닌 두 사건 $A, B$에 대하여 다음이 성립한다.$$\pr(A\cap B) = \pr(B\,|\,A)\pr(A) = \pr(A\,|\,B)\pr(B)$$}

\prob{장난감 100개 중 20개가 불량품이다. 이 중 2개를 임의로 추출할 때, 2개 모두 불량품일 확률을 구하여라.}

\defn{두 사건 $A, B$에 대하여 사건 $A$가 일어났을 때의 사건 $B$의 조건부확률이 사건 $B$가 일어날 확률과 같을 때, 즉 $$\pr(B\,|\,A) = \pr(B\,|\,A^C) = \pr(B)$$ 이면, 두 사건 $A, B$는 서로 \textbf{독립}(independent)이라 하고, 기호로 $A\indep B$ 와 같이 나타낸다. 두 사건이 독립이 아닐 때는 \textbf{종속}이라 한다.}

\prob{주사위 2개를 던질 때, 다음 두 사건이 독립인지 판정하여라.
\begin{enumerate}
	\item[(1)] $A$: 두 주사위 눈의 합이 6인 사건, $B$: 첫 번째 주사위 눈이 4인 사건
	\item[(2)] $A$: 두 주사위 눈의 합이 7인 사건, $B$: 첫 번째 주사위 눈이 4인 사건
\end{enumerate}
}

\thm{공사건이 아닌 두 사건 $A, B$에 대하여 다음 조건은 서로 동치이다.
\begin{enumerate}
	\item[(1)] $A, B$가 서로 독립이다. \vspace{-3mm}
	\item[(2)] $\pr(A\cap B) = \pr(A)\pr(B)$
\end{enumerate}}

\thm{공사건이 아닌 두 사건 $A, B$에 대하여 다음이 성립한다.
\begin{center}
	[$A, B$가 독립] $\iff$ [$A^C, B$가 독립] $\iff$ [$A, B^C$가 독립] $\iff$ [$A^C, B^C$가 독립]
\end{center}}

\prob{다음을 증명하여라.
\begin{enumerate}
	\item[(1)] 정리 5.8.
	\item[(2)] 두 사건 $A, B$가 서로 배반사건이면, 사건 $A, B$는 종속이다. 
\end{enumerate}
}

\prob{두 사건 $A, B$가 서로 독립이고 $\pr(A)=0.25, \pr(B)=0.4$ 일 때, 다음을 구하여라.
\begin{enumerate}
	\item[(1)] $\pr(A\cap B)$
	\item[(2)] $\pr(A^C\cap B)$
	\item[(3)] $\cndpr{A}{B^C}$
	\item[(4)] $\cndpr{B^C}{A^C}$
\end{enumerate}
}

\prob{두 사건 $A, B$가 서로 독립이고, $\pr(A\cup B) =0.8, \pr(A\cap B)=0.3$ 일 때, $\pr(A), \pr(B)$를 각각 구하여라. (단, $\pr(A)>\pr(B)$)}

\defn{공사건이 아닌 사건 $A_1, \dots, A_n$에 대하여, $$\pr(A_i\cap A_j) = \pr(A_i)\pr(A_j) \quad \text{for all }1\leq i\neq j\leq n$$
	이 성립하면 사건 $A_1, \dots, A_n$이 \textbf{쌍마다 독립}(pairwise independent)이라고 한다. }

\defn{공사건이 아닌 사건 $A_1, \dots, A_n$가 있다. 임의의 $J\subset \{1, 2,\dots, n\}$ 에 대해
	$$\pr\left(\bigcap_{i\,\in\, J} A_i\right) = \prod_{i\, \in\, J} \pr(A_i)$$
	이 성립하면 사건 $A_1, \dots, A_n$이 \textbf{상호 독립}(mutually independent)이라고 한다. }

\prob{사건 $A, B, C$에 대하여 $\pr(A\cap B\cap C) = \pr(A)\pr(B)\pr(C)$ 이지만 $A, B, C$ 가 상호 독립은 아닌 $A, B, C$의 예시를 찾아라.}\\

\defn{동일한 시행을 반복할 때, 각 시행에서 일어나는 사건이 서로 독립이면 이러한 시행을 \textbf{독립시행}이라고 한다.}

\thm{\textbf{(독립시행의 확률)} $1$회의 시행에서 사건 $A$가 일어날 확률을 $p$ 라 할 때, 이 시행을 $n$회 반복하는 독립시행에서 사건 $A$가 $r$번 일어날 확률은 다음과 같다. $${n\choose r} p^r(1-p)^r \quad \big(r=0, 1, \dots, n\big)$$}

\prob{주사위를 한 번 던지는 시행에서 3의 배수의 눈이 나오는 사건을 $A$라 할 때, 다음 물음에 답하여라.
\begin{enumerate}
	\item[(1)] $\pr(A)$를 구하여라.
	\item[(2)] 주사위를 20번 던지는 시행에서 사건 $A$가 12번 일어날 확률을 구하여라.
	\item[(3)] 주사위를 100번 던지는 시행에서 사건 $A$가 $r$번 일어날 확률을 구하여라.
\end{enumerate}
}

\thm{(\textbf{전확률공식} - Law of Total Probability) 표본공간 $S$의 분할인 사건 $A_1, \dots, A_n$에 대하여 다음이 성립한다.
$$\pr(B) = \sum_{i=1}^n \pr(B\, |\, A_i)\,\pr(A_i)$$
\textbf{증명.} $\ds \pr(B) = \pr\left( \bigcup_{i=1}^n\left(B\cap A_i\right)\right) = \sum_{i=1}^n \pr(B\cap A_i) = \sum_{i=1}^n\pr(B\, |\, A_i)\,\pr(A_i) $ \qed
}

\prob{H 대학의 통계학과 학생의 30\%는 1학년이고, 25\%는 2학년이고, 25\%는 3학년이고, 20\%는 4학년이라고 하자. 그런데 1학년의 50\%, 2학년의 30\%, 3학년의 10\%, 4학년의 2\%가 수학 과목의 수강생이라 한다. 통계학과 학생 중 한 학생을 임의로 선택할 때 그 학생이 수학 과목의 수강생일 확률을 구하여라.}\\

\thm{(\textbf{베이즈 정리} - Bayes' Theorem) 사건 $A_1, \dots, A_n$이 표본공간 $S$의 분할이고 $\pr(B)>0$ 이면 다음이 성립한다.
$$\pr(A_k \, |\, B) = \frac{\pr(B\,|\,A_k)\,\pr(A_k)}{\ds\sum_{i=1}^n \pr(B\, |\, A_i)\,\pr(A_i)} \qquad (k=1, \dots, n)$$
\textbf{증명.} 조건부확률의 정의와 전확률공식으로부터 자명하다. \qed
}

\prob{베이즈 정리를 증명하여라.}\\\\

\prob{주머니 A에는 흰 공 3개, 검은 공 2개가 들어 있고, 주머니 B에는 흰 공 2개, 검은 공 3개가 들어 있다. 임의로 주머니 하나를 택하여 2개의 공을 동시에 꺼냈더니 모두 흰 공이었을 때, 그것이 주머니 B에서 나왔을 확률을 구하여라.}\\

\prob{A 주머니에 흰 공 2개, 검은 공 5개, B 주머니에 흰 공 3개, 검은 공 4개가 들어있다. A 주머니에서 한 개의 공을 임의로 꺼내어 B 주머니에 넣은 다음 다시 B 주머니에서 하나의 공을 꺼내기로 한다. B에서 꺼낸 공이 흰 공일 때, A에서 B로 옮겨진 공이 흰 공이었을 확률을 구하여라.}\\

\prob{어떤 지역의 결핵환자의 비율이 0.1\%로 알려져 있다. 결핵에 걸려있는지 알아보는 검사에서 결핵에 걸렸을 때 양성 반응이 나타날 확률은 95\%이고 그렇지 않을 때 양성 반응이 나타날 확률은 1.1\%라고 한다. 양성 반응이 나타났을 때 결핵에 걸렸을 확률을 구하여라.}


\pagebreak



\end{document}
\documentclass[12pt]{article}
\usepackage{graphicx}

\usepackage{kotex}
\usepackage{amsmath}
\usepackage{amsfonts}
\usepackage{amssymb}
\usepackage{mathtools}
\usepackage{geometry}
\usepackage{amsthm}
\usepackage{pgfplots}
\usepackage{graphicx}
\usepackage{tikz}
\usepackage{xcolor}
\pgfplotsset{compat=1.7}
\pgfmathdeclarefunction{gauss}{2}{\pgfmathparse{1/(#2*sqrt(2*pi))*exp(-((x-#1)^2)/(2*#2^2))}%
}
\graphicspath{ {./images/} }
\geometry{
	top = 30mm,
	left = 25mm,
	right = 25mm,
	bottom = 30mm
}
\geometry{a4paper}

\usepackage{hyperref}
\hypersetup{
    % hidelinks,
    colorlinks=true,
    linkcolor=black,
    unicode=true,
    urlcolor=blue,
    citecolor=blue,
}

\renewcommand{\baselinestretch}{1.4}

\theoremstyle{definition}
\newtheorem{theorem}{\sffamily 정리}[section]
\theoremstyle{definition}
\newtheorem{definition}[theorem]{\sffamily 정의}
\newtheorem{problem}[theorem]{\sffamily 연습문제}

\newcommand{\defn}[1]{\begin{definition}#1\end{definition}~}
\newcommand{\thm}[1]{\begin{theorem}#1\end{theorem}~}
\newcommand{\prob}[1]{\begin{problem}#1\end{problem}~}

\usepackage{titlesec}

\title{\textbf{Introductory Statistics}}
\author{Sungchan Yi}
\date{January 2019}

\titleformat*{\section}{\large\bfseries}

\newcommand{\p}[2]{_{#1}\text{P}_{#2}}
\newcommand{\ds}{\displaystyle}
\newcommand{\pr}{\text{P}}
\newcommand{\cndpr}[2]{\pr\!\left(#1\,|\,#2\right)}
\newcommand{\indep}{\mathrel{\!\perp \!\!\!\perp\!}}
\newcommand{\ex}{\mathbf{E}}
\newcommand{\var}{\mathbf{V}}
\newcommand{\bi}{\mathbf{B}}
\newcommand{\mc}[1]{\mathcal{#1}}
\newcommand{\mr}[1]{\mathrm{#1}}
\newcommand{\chisq}{\chi^2}
\renewcommand{\qed}{\hfill\ensuremath{\square}}

\begin{document}
\maketitle

\tableofcontents
\pagebreak

\section{자료의 생성}
\textbf{통계학}(statistics)이란, 주어진 문제에 대하여 합리적인 답을 줄 수 있도록 숫자로 표시되는 정보를 \textbf{수집}하고 정리하며, 이를 해석하고 \textbf{신뢰성 있는 결론}을 이끌어 내는 방법을 연구하는 과학의 한 분야이다.\\\\
그러면 두 가지 질문이 생긴다.
\begin{enumerate}
	\item \textbf{수집}: 어떻게 수집해야 전체를 잘 대표할 수 있는가?
	\item \textbf{신뢰성 있는 결론}: 어떻게 신뢰성을 측정하여 결론을 내릴 것인가?
\end{enumerate}
\vspace{3mm}
\defn{~
\begin{itemize}
	\item \textbf{추출단위}(sampling unit): 전체를 구성하는 각 개체
	\item \textbf{특성값}(characteristic): 각 추출단위의 특성을 나타내는 값
	\item \textbf{모집단}(population): 관심의 대상이 되는 모든 추출단위의 특성값을 모아 놓은 것\\ 추출단위의 개수가 유한하면 \textbf{유한모집단}, 무한하면 \textbf{무한모집단}이라 한다.
	\item \textbf{표본}(sample): 실제로 관측한 추출단위의 특성값의 모임
\end{itemize}
}

\defn{(자료의 종류)
	\begin{enumerate}
		\item \textbf{범주형 자료}(categorical data), \textbf{질적 자료}(qualitative data)는 관측 결과가 몇 개의 범주 또는 항목의 형태로 나타나는 자료이다.
		\begin{itemize}
			\item \textbf{명목자료}(nominal data): 순위의 개념이 없다. 예) 혈액형, 성별
			\item \textbf{순서자료}(ordinal data): 순위의 개념을 갖는다. 예) A $\sim$ F 학점, 9등급제
		\end{itemize}
		\item \textbf{수치형 자료}(numerical data), \textbf{양적 자료}(quantitative data)는 자료 자체가 숫자로 표현되며 숫자 자체가 자료의 속성을 반영한다.
		\begin{itemize}
			\item \textbf{이산형 자료}(discrete data) 예) 교통사고 건수
			\item \textbf{연속형 자료}(continuous data) 예) 키, 몸무게
		\end{itemize}
	\end{enumerate}}

\defn{(통계학의 분류)
\begin{itemize}
	\item \textbf{기술통계학}(descriptive statistics)은 표나 그림 또는 대푯값 등을 통하여 수집된 자료의 특성을 쉽게 파악할 수 있도록 자료를 정리$\cdot$요약 하는 방법을 다루는 분야이다.
	\item \textbf{추측통계학}(inferential statistics)은 표본에 내포된 정보를 분석하여 모집단의 여러가지 특성에 대하여 과학적으로 추론하는 방법을 다루는 분야이다.
\end{itemize}
}

\defn{$N$개의 추출단위로 구성된 유한모집단에서 $n$개의 추출단위를 비복원추출할 때, $_N \mathrm{C}_n$개의 모든 가능한 표본들이 동일한 확률로 추출되는 방법을 \textbf{단순랜덤추출법}(simple random sampling)이라 하고, 이 방법을 위해서는 난수표(random number table)나 난수생성기(random number generator) 등을 이용한다. 그리고 단순랜덤추출로 얻은 표본을 \textbf{단순랜덤표본}(simple random sample)이라 한다.
}

\defn{(통계적 실험)
\begin{itemize}
	\item 실험이 행해지는 개체를 \textbf{실험단위}(experimental unit/subject)라 하고, 각각의 실험단위에 특정한 실험환경 또는 실험조건을 가하는 것을 \textbf{처리}(treatment)라 한다.
	\item 처리를 받는 집단을 \textbf{처리집단}(treatment group), 처리를 받지 않은 집단을 \textbf{대조집단}(control group)이라 한다.
	\item 실험환경이나 실험조건을 나타내는 변수를 \textbf{인자}(factor)라 하고, 인자가 취하는 값을 그 인자의 \textbf{수준}(level)이라 한다.
	\item 인자에 대한 반응을 나타내는 변수를 \textbf{반응변수}(response variable)라 한다.
	\item 실험단위가 처리집단이나 대조집단에 들어갈 기회를 동등하게 부여하는 방법을 \textbf{랜덤화}(randomization)라 한다.
	\item 랜덤화에 의해 모든 실험단위를 각 처리에 배정하는 실험계획을 \textbf{완전 랜덤화 계획}(completely randomized design)이라 한다.
	\item 실험 이전에 동일 처리에 대한 반응이 유사할 것으로 예상되는 실험단위들끼리 모은 것을 \textbf{블록}(block)이라 하고, 랜덤화에 의해 모든 블록을 각 처리에 배정하는 실험계획을 \textbf{블록화}(randomized block design)라 한다.
\end{itemize}
}

\defn{\textbf{(통계적 실험계획의 원칙)}
\begin{enumerate}
	\item \textbf{대조}(control): 관심 인자 이외의 다른 외부 인자의 효과를 극소화하고, 처리에 대한 대조집단을 통해 비교 실험을 한다.
	\item \textbf{랜덤화}(randomization): 완전랜덤화계획
	\item \textbf{반복 시행}(replication): 처리효과의 탐지를 용이하게 하기 위해 반복 시행한다.
\end{enumerate}
}


\pagebreak

\section{대푯값과 산포도}
주어진 자료의 변량 $x_1, \cdots, x_n$에 대하여
\defn{(산술)\textbf{평균}(mean $\mu$)은 변량의 총합을 변량의 개수로 나눈 값이다. $$\mu = \frac{1}{n}\sum_{i=1}^nx_i$$}

\defn{변량 $x_i$의 \textbf{편차}는 $x_i-\mu$ 로 정의한다. [편차의 제곱]의 평균을 \textbf{분산}(variance $\sigma^2$)으로, 분산의 양의 제곱근을 \textbf{표준편차}(standard deviation $\sigma$)로 정의한다. $$\sigma^2 = \frac{1}{n}\sum_{i=1}^n(x_i-\mu)^2$$}

\thm{$\ds \sigma^2 = \frac{1}{n}\sum_{i=1}^n x_i^2 - \mu^2$ \quad $\big($분산은 [변량$^2$의 평균] - [평균의 제곱]$\big)$}

\defn{모집단의 평균을 \textbf{모평균}(population mean), 모집단의 분산을 \textbf{모분산}(population variance), 모집단의 표준편차를 \textbf{모표준편차}(population standard deviation)라고 한다.
}

\defn{특성값을 작은 것부터 순서대로 나열했을 때 $p$\%의 특성값이 그 값보다 작거나 같고, $(100-p)$\%의 특성값이 그 값보다 크거나 같게 되는 값을 \textbf{제 $p$ 백분위수}($p$-th percentile)라 한다.}

\defn{(사분위수 - quartile)
\begin{itemize}
	\item \textbf{제 1 사분위수}(first quartile): 제 25 백분위수이며, $Q_1$ 으로 표기한다. 
	\item \textbf{제 2 사분위수}(second quartile) 또는 \textbf{중앙값}(median): 제 50 백분위수이며, $Q_2$ 으로 표기한다. 
	\item \textbf{제 3 사분위수}(third quartile): 제 75 백분위수이며, $Q_3$ 으로 표기한다.
\end{itemize}	
}

\defn{자료의 값들 중 가장 자주 등장하는 값을 \textbf{최빈값}(mode)라고 한다. 최빈값은 유일하지 않을 수도 있다.}

\defn{변량과 중앙값 사이의 거리에 대한 평균을 \textbf{평균절대편차}(mean absolute deviance $MAD$)라 한다. $$MAD = \frac{1}{n}\sum_{i=1}^n \left|x_i - Q_2\right|$$} 

\defn{변량 $x_i$가 \textbf{최댓값}(maximum)이면 모든 $j$에 대해 $x_i\geq x_j$ 이고, $x_i$가 \textbf{최솟값}(minimum)이면 모든 $j$에 대해 $x_i\leq x_j$ 이다. 최댓값에서 최솟값을 뺀 값을 \textbf{범위}(range $R$)라 한다.}


\defn{$Q_3$에서 $Q_1$을 뺀 값을 \textbf{사분위수범위}(interquartile range $IQR$)로 정의한다. $$IQR = Q_3-Q_1$$}\\
위 대푯값들을 표본에 대해서도 정의할 수 있다. 모집단으로부터 표본 $x_1, \dots, x_n$를 얻었다고 하고, 이를 오름차순으로 나열한 것을 $x_{(1)}, \dots, x_{(n)}$이라 하자.\\

\defn{~
\begin{itemize}
	\item \textbf{표본평균}(sample mean): $\ds \overline{x} = \frac{1}{n} \sum_{i=1}^n x_i$
	\item \textbf{표본분산}(sample variance): $\ds s^2=\frac{1}{n-1}\sum_{i=1}^n (x_i-\overline{x})^2$
	\item \textbf{표본표준편차}(sample standard deviation): $s = \sqrt{s^2}$
\end{itemize}
}

\defn{표본을 크기 순서로 나열했을 때 $p$\%가 그 값보다 작고, $(100-p)$\%가 그 값보다 크게 되는 값을 \textbf{표본의 제 $p$ 백분위수}라 하고, 다음과 같이 계산한다.
$$\begin{cases}
	\dfrac{x_{(k)}+x_{(k+1)}}{2} & \text{ if  }\: n\cdot \dfrac{p}{100} = k\\x_{(k+1)} & \text{ if  }\: k < n\cdot \dfrac{p}{100} < k+1
\end{cases}$$
}

\defn{\textbf{표본의 제 $i$ 사분위수} $\widehat{Q_i}$ 는 표본의 제 $25i$ 백분위수로 정의한다. (단, $i=1, 2, 3$)}

\defn{~
\begin{itemize}
	\item \textbf{표본의 평균절대편차}: $\ds \widehat{MAD} = \frac{1}{n}\sum_{i=1}^n \left|x_i-\widehat{Q_2}\right|$
	\item \textbf{표본의 범위}: $\widehat{R} = x_{(n)}-x_{(1)}$
	\item \textbf{표본의 사분위수범위}: $\widehat{IQR} = \widehat{Q_3} - \widehat{Q_1}$
\end{itemize}
}
\pagebreak

\input{./sec/03.tex}
\input{./sec/04.tex}
\input{./sec/05.tex}
\input{./sec/06.tex}
\input{./sec/07.tex}
\input{./sec/08.tex}
\section{표본분포}
모집단 전체를 조사하는 것은 비용이 많이 들고, 또 현실적으로 어렵다. 따라서 통계적 추정을 할 때에는 표본을 뽑아 조사하는 것이 경제적이다.\\

\defn{모집단에서 임의추출한 표본으로부터 얻은 통계량은 확률변수이므로 분포를 가지게 된다. 이를 \textbf{표본분포}(sampling distribution)라 한다.}

\defn{모집단에서 임의추출한 크기 $n$인 표본을 $X_1, \dots, X_n$ 이라 할 때,
\begin{itemize}
	\item \textbf{표본평균}(sample mean): $\overline{X} = \ds \frac{1}{n}\sum_{i=1}^n X_i$
	\item \textbf{표본분산}: $S^2 = \ds \frac{1}{n-1}\sum_{i=1}^n \left(X_i-\overline{X}\right)^2$
\end{itemize}
표본을 뽑을 때마다 표본평균, 표본분산은 달라질 수 있으므로 $\overline{X}, S^2$은 확률변수가 된다. 따라서, 기댓값, 분산, 표준편차도 계산할 수 있다.
}

\prob{모집단 $\{1, 3, 5, 7\}$ 에서 크기가 2인 표본을 복원추출 할 때, 표본평균 $\overline{X}$의 확률분포가 다음과 같다.
\begin{center}
	\begin{tabular}{c|c|c|c|c|c|c|c|c}
		$\overline{X}$ & 1&2&3&4&5&6&7&합계\\\hline
		$\pr(\overline{X} = \overline{x})$ & $\frac{1}{16}$ & $\frac{1}{8}$ & $a$ & $b$ & $\frac{3}{16}$ & $c$ & $\frac{1}{16}$ &1

	\end{tabular}
\end{center}
이 때, $a, b, c$ 의 값을 구하고, 확률변수 $\overline{X}$의 기댓값과 분산을 구하여라.}\\

\defn{$X_i$ 를 $i$-번째로 뽑힌 추출단위의 특성값을 나타내는 확률변수라 하자. 다음 조건을 만족하는 $X_1, \dots, X_n$ 을 \textbf{랜덤표본}(random sample)이라 한다.
\begin{enumerate}
	\item[(1)] (유한모집단) 단순랜덤 비복원추출로 뽑은 표본
	\item[(2)] (무한모집단) $X_i$ 들은 서로 독립이고 각 분포가 모집단 분포와 동일
\end{enumerate}
참고: 유한모집단에서 모집단의 크기가 큰 경우에는 흔히 무한모집단에서의 랜덤표본으로 간주하여 표본분포를 구한 다음 이를 실제표본분포의 근사분포로 사용한다.
}

\thm{모평균이 $\mu$이고 모표준편차가 $\sigma$인 모집단에서\footnote{무한모집단의 경우 복원추출이나 비복원추출이나 큰 차이가 없다.} 복원추출하여 뽑은 크기가 $n$인 표본의 표본평균 $\overline{X}$에 대하여 다음이 성립한다. $$\ex(\overline{X}) = \mu, \: \var(\overline{X}) = \frac{\sigma^2}{n}, \: \sigma(\overline{X}) = \frac{\sigma}{\sqrt{n}}$$
\textbf{증명}. $\overline{X} = \ds \frac{1}{n} \sum_{i=1}^n X_i$ 를 이용한다.\\
$$\ex(\overline{X}) = \ds \frac{1}{n}\ex(X_1+\cdots + X_n) = \frac{n\mu}{n} = \mu$$$$\var(\overline{X}) =\ds \frac{1}{n^2}\var(X_1+\cdots+X_n) = \frac{1}{n^2}\sum_{i=1}^n \var(X_i) = \frac{n\sigma^2}{n^2} = \frac{\sigma^2}{n}$$ \qed\\
\textbf{참고}: 모평균이 $\mu$, 모분산이 $\sigma^2$, 크기가 $N$인 유한모집단에서 크기 $n$인 표본을 비복원추출하는 경우에는 다음이 성립한다. $$\ex(\overline{X}) = \mu, \: \var(\overline{X}) = \frac{N-n}{N-1}\cdot\frac{\sigma^2}{n}$$
여기서 $\sqrt{\frac{N-n}{N-1}}$ 은 finite-population correction factor (FPC) 로 불린다.
}


\prob{연습문제 9.3 에서 구한 $\overline{X}$ 의 기댓값과 분산이 위 정리를 사용하여 구한 것과 일치함을 확인하여라.}

\thm{표본의 크기가 클수록 표본평균과 모평균의 오차가 줄어든다.}

\thm{모집단의 분포가 $\mc{N}(\mu, \sigma^2)$ 일 때, 표본평균 $\overline{X}$는 $\mc{N}(\mu, \sigma^2/n)$ 을 따른다.}

\prob{$\mc{N}(100, 2^2)$ 을 따르는 모집단에서 크기가 4인 표본을 임의추출할 때, 표본평균 $\overline{X}$가 따르는 분포를 구하여라.}

\prob{어느 전기 회사에서 생산하는 전구의 수명을 나타내는 확률변수 $X$에 대하여, $X\sim \mc{N}(2000, 200^2)$ 이라고 한다. 이 회사가 생산한 전구 중 임의추출한 $n$개 전구의 평균 수명을 $\overline{X}$라 할 때, $\pr(1900 \leq \overline{X}\leq 2100)\geq0.9$ 가 성립하기 위한 $n$의 최솟값을 구하여라. 단, $z_{0.05} = 1.65$ 이다.}

\thm{(\textbf{중심극한정리} - Central Limit Theorem) 평균이 $\mu$이고 분산이 $\sigma^2$인 임의의 무한모집단에서 표본의 크기 $n$이 충분히 크면, 랜덤표본의 표본평균 $\overline{X}$는 근사적으로 정규분포 $\mc{N}\left(\mu, \ds \frac{\sigma^2}{n}\right)$ 을 따른다.\footnote{당연히, $n$이 클수록 근사는 정확해진다.}}

\prob{대학 신입생 신장의 평균이 168cm이고 표준편차가 6cm임이 알려져 있다. 100명의 신입생을 단순랜덤추출하는 경우 표본평균이 167cm 이상 169cm 이하일 확률을 구하여라. 단, $z_{0.0475} = 1.67$ 이다.}\\\\

\defn{확률변수 $Z_1, \dots, Z_k$ 이 $\mc{N}(0, 1)$의 랜덤표본일 때, $V = Z_1^2+\cdots+Z_k^2$ 의 분포를 \textbf{자유도}(degrees of freedom) $k$인 \textbf{카이제곱분포}($\chi^2$-distribution)라고 한다. 기호로는 다음과 같이 나타낸다. $$Z_1^2+\cdots+Z_k^2 = V \sim \chi^2(k)$$그리고 확률밀도함수는 다음과 같다.\footnote{$\Gamma$ 는 감마함수(Gamma function)을 나타내는 기호이다.} $$\frac{1}{2^{k/2}\Gamma(k/2)}x^{k/2-1}e^{-x/2} \qquad (x > 0)$$}

\begin{center}
	\includegraphics[width=10.4cm, height=7cm]{chisq.png}\\
	$\chi^2$-분포의 형태 ($k$: 자유도)
\end{center}

\defn{$V\sim \chi^2(k)$ 일 때, $\pr(V>v)=\alpha$ 인 $v$의 값을 $\chi^2_\alpha(k)$ 로 정의한다.}

\thm{(카이제곱분포의 가법성) $V_1, V_2$가 서로 독립이면 다음이 성립한다.
\begin{enumerate}
	\item[(1)] $V_1\sim \chisq(k_1)$, $V_2\sim \chisq(k_2)$ 이면 $V_1+V_2 \sim \chisq(k_1+k_2)$
	\item[(2)] $V_1\sim \chisq(k_1)$, $V_1+V_2\sim \chisq(k_1+k_2)$ 이면 $V_2\sim \chisq(k_2)$
\end{enumerate}}

\thm{$X_1, \dots, X_n$이 $\mc{N}(\mu, \sigma^2)$의 랜덤표본일 때, 표본분산 $S^2 = \ds \frac{1}{n-1}\sum_{i=1}^n (X_i-\overline{X})^2$ 에 대하여 다음이 성립한다. $$\frac{(n-1)S^2}{\sigma^2} \sim \chisq(n-1)$$
\textbf{증명}. 모든 $i$에 대해 $\dfrac{X_i-\mu}{\sigma} \sim \mc{N}(0, 1)$ 이고 서로 독립이므로 카이제곱분포의 정의로부터
$$\sum_{i=1}^n \left(\frac{X_i-\mu}{\sigma}\right)^2 \sim \chisq(n)$$
이 성립한다. 그런데, 표본분산 $S^2$의 정의로부터
$$\sum_{i=1}^n \left(\frac{X_i-\mu}{\sigma}\right)^2 = \sum_{i=1}^n \left(\frac{X_i-\overline{X}}{\sigma}\right)^2 + n\left(\frac{\overline{X}-\mu}{\sigma}\right)^2 = \frac{(n-1)S^2}{\sigma^2} + \left(\frac{\overline{X}-\mu}{\sigma/\sqrt{n}}\right)^2$$
이고 $\ds \left(\frac{\overline{X}-\mu}{\sigma/\sqrt{n}}\right)^2 \sim \chisq(1)$ 이므로 가법성에 의해 $\ds \frac{(n-1)S^2}{\sigma^2} \sim \chisq(n-1)$.\qed
}

\thm{분산이 동일한 두 정규모집단 $\mc{N}(\mu_1, \sigma^2)$, $\mc{N}(\mu_2, \sigma^2)$ 에서 각각 뽑은 랜덤표본 $X_1, \dots, X_{n_1}$ 과 $Y_1, \dots, Y_{n_2}$ 이 서로 독립이라고 하자. $S_1^2, S_2^2$를 각각 $X_i, Y_i$ 의 표본분산이라 할 때, \textbf{합동표본분산}(pooled sample variance) $$S_p^2 = \frac{(n_1-1)S_1^2 +(n_2-1)S_2^2}{n_1+n_2-2}$$에 대하여 다음이 성립한다. $$\frac{(n_1+n_2-2)S_p^2}{\sigma^2} \sim \chisq (n_1+n_2-2)$$
\textbf{증명}. 다음을 이용하면 가법정리에 의해 자명하다. $$S_p^2 = \frac{(n_1-1)}{(n_1-1)+(n_2-1)} S_1^2+\frac{(n_2-1)}{(n_1-1)+(n_2-1)} S_2^2$$\qed}

\defn{$Z\sim \mc{N}(0, 1)$ 과 이와 독립인 확률변수 $V$가 자유도가 $k$인 카이제곱분포를 따를 때, $T=\dfrac{Z}{\sqrt{V/k}}$ 의 분포를 자유도가 $k$인 \textbf{$t$-분포}($t$-distribution)라 한다.\footnote{스튜던트의(Student's) $t$-분포 라고 불리기도 한다.} 기호로는 다음과 같이 나타낸다. $$\frac{Z}{\sqrt{V/k}} = T \sim t(k)$$그리고 확률밀도함수는 다음과 같다.
	$$\frac{\Gamma\left(\frac{k+1}{2}\right)}{\sqrt{k\pi}\,\Gamma\left(\frac{k}{2}\right)}\left(1+\frac{x^2}{k}\right)^{-\frac{k+1}{2}} \qquad (-\infty < x <\infty)$$
}
\begin{center}
	\includegraphics[width=9.0625cm, height=5cm]{tdist}\\
	$t$-분포의 형태 ($n$: 자유도)
\end{center}

\thm{$t$-분포 곡선은 0을 중심으로 좌우 대칭의 밀도곡선을 가지며, 자유도 $k$가 커지면 $\mc{N}(0, 1)$과 비슷하며, 일반적으로 표준정규분포보다 더 두꺼운 꼬리를 갖고 있다.}

\defn{$T\sim t(k)$ 일 때, $\pr(T>t)=\alpha$ 인 $t$의 값을 $t_\alpha(k)$ 로 정의한다.}

\thm{$X_1, \dots, X_n$ 이 $\mc{N}(\mu, \sigma^2)$의 랜덤표본일 때, 다음이 성립한다. $$\frac{\overline{X}-\mu}{S/\sqrt{n}}\sim t(n-1)$$
\textbf{증명}. 다음 변형을 이용한다.
$$\frac{\overline{X}-\mu}{S/\sqrt{n}} = \frac{\left(\overline{X}-\mu\right) / \left(\dfrac{\sigma}{\sqrt{n}}\right)}{\sqrt{\dfrac{(n-1)S^2}{\sigma^2}/ (n-1)}}$$
$\ds \frac{\overline{X}-\mu}{\sigma/\sqrt{n}} \sim \mc{N}(0, 1)$ 이고, $\dfrac{(n-1)S^2}{\sigma^2} \sim \chisq(n-1)$ 이며, $\overline{X}$ 와 $S^2$은 서로 독립임이 알려져 있다.\footnote{일반적으로는 독립이 아니다. 이 정리의 가정 하에서만 독립이다. See Basu's Theorem.} $t$-분포의 정의에 의해 성립한다.\qed
}

\thm{분산이 동일한 두 정규모집단 $\mc{N}(\mu_1, \sigma^2)$, $\mc{N}(\mu_2, \sigma^2)$ 에서 각각 뽑은 랜덤표본 $X_1, \dots, X_{n_1}$ 과 $Y_1, \dots, Y_{n_2}$ 이 서로 독립이라고 하자. $X_i, Y_i$ 의 표본분산 $S_1^2, S_2^2$, 합동표본분산 $S_p^2$에 대하여 다음이 성립한다. $$\frac{(\overline{X}-\overline{Y}) - (\mu_1-\mu_2)}{S_p \sqrt{\dfrac{1}{n_1}+\dfrac{1}{n_2}}} \sim t(n_1+n_2-2)$$
\textbf{증명}. 다음 변형을 이용한다.
$$\frac{(\overline{X}-\overline{Y}) - (\mu_1-\mu_2)}{S_p \sqrt{\dfrac{1}{n_1}+\dfrac{1}{n_2}}} = \frac{\left[(\overline{X}-\overline{Y}) - (\mu_1-\mu_2)\right] / \left(\sigma \sqrt{\dfrac{1}{n_1} + \dfrac{1}{n_2}}\right)}{\sqrt{\dfrac{(n_1+n_2-2)S_p^2}{\sigma^2}/(n_1+n_2-2)}}$$
분모는 $\mc{N}(0, 1)$ 을 따르고, 정리 9.17에 의해 $t$-분포의 정의를 만족한다. \qed
}

\defn{$V_1\sim \chisq(k_1)$, $V_2\sim \chisq(k_2)$ 이고 $V_1, V_2$가 서로 독립일 때, $$F=\frac{V_1/k_1}{V_2/k_2}$$ 의 분포를 자유도 $(k_1, k_2)$ 인 $F$-분포라고 한다. 기호로는 다음과 같이 나타낸다.$$\frac{V_1/k_1}{V_2/k_2}=F\sim F(k_1, k_2)$$
그리고 확률밀도함수는 다음과 같다.\footnote{B 는 베타함수(Beta function)를 나타낸다.}
$$\frac{\sqrt{\dfrac{(k_1x)^{k_1}k_2^{k_2}}{(k_1x+k_2)^{k_1+k_2}}}}{x\mr{B}\left(\dfrac{k_1}{2}, \dfrac{k_2}{2}\right)} \qquad (x > 0)$$}

\defn{$F\sim F(k_1, k_2)$ 일 때, $\pr(F>f)=\alpha$ 인 $f$의 값을 $F_\alpha(k_1, k_2)$ 로 정의한다.}

\thm{~
\begin{enumerate}
	\item[(1)] $F\sim F(k_1, k_2)$ 이면, $1/F \sim F(k_2, k_1)$ 이다.
	\item[(2)] $F_{1-\alpha}(k_2, k_1) = 1/F_\alpha(k_1, k_2)$
	\item[(3)] $T\sim t(k)$ $\iff$ $T^2 \sim F(1, k)$
\end{enumerate}
}

\thm{두 정규모집단 $\mc{N}(\mu_1, \sigma_1^2)$, $\mc{N}(\mu_2, \sigma_2^2)$ 에서 각각 뽑은 랜덤표본 $X_1, \dots, X_{n_1}$ 과 $Y_1, \dots, Y_{n_2}$ 이 서로 독립이라고 하자. $X_i, Y_i$ 의 표본분산 $S_1^2, S_2^2$에 대하여 다음이 성립한다.
$$\frac{S_1^2/\sigma_1^2}{S_2^2/\sigma_2^2} \sim F(n_1-1, n_2-1)$$
\textbf{증명}. 다음과 같이 변형한다.
$$\frac{S_1^2/\sigma_1^2}{S_2^2/\sigma_2^2} = \frac{\dfrac{(n_1-1)S_1^2}{\sigma_1^2}/(n_1-1)}{\dfrac{(n_2-2)S_2^2}{\sigma_2^2}/(n_2-1)}$$
$\dfrac{(n_1-1)S_1^2}{\sigma_1^2} \sim \chisq(n_1-1)$, $\dfrac{(n_2-2)S_2^2}{\sigma_2^2}\sim \chisq(n_2-1)$ 이므로 $F$-분포의 정의를 만족한다. \qed
}

\pagebreak

\section{통계적 추론}
\defn{표본으로부터의 정보를 이용하여 모집단에 관한 추측이나 결론을 이끌어내는 과정을 \textbf{통계적 추론}(statistical inference)이라 한다. 모집단의 특성치(모수)에 대한 추측값을 제공하고 그 오차의 한계를 제시하는 과정을 \textbf{추정}(estimation)이라 하고, 다음 두 가지 종류가 있다.
\begin{itemize}
	\item \textbf{점추정}(point estimation): 모수의 참값이라고 추측되는 하나의 추정값을 제공
	\item \textbf{구간추정}(interval estimation): 모수의 참값이 속할 것으로 기대되는 범위를 추측
\end{itemize}
}

\defn{정의 및 표기법
\begin{itemize}
	\item \textbf{모수}(population parameter) $\theta$: 모집단의 특성을 나타내는 수치적 측도
	\item 랜덤표본은 $X_1, \dots, X_n$, 랜덤표본의 관측값은 $x_1, \dots, x_n$ 으로 표기한다.
	\item \textbf{추정량}(estimator): 미지의 모수 $\theta$의 추정에 사용되는 통계량으로, $\hat{\theta}(X_1, \dots, X_n)$ 혹은 $\hat{\theta}$으로 표기한다.
	\item \textbf{추정값}(estimate): 추정량 $\hat{\theta}(X_1, \dots, X_n)$의 관측값으로, $\hat{\theta}(x_1, \dots, x_n)$로 표기한다.
\end{itemize}
추정량과 추정값의 관계는 확률변수와 그 관측값의 관계이다.\\
모평균 $\mu$를 추정하기 위해 추정량으로는 표본평균 $\hat{\mu} = \overline{X} = \frac{1}{n}\sum_{i=1}^n X_i$ 을 사용하고, 그 추정값으로는 관측한 결과인 $\hat{\mu} = \overline{x} = \frac{1}{n}\sum_{i=1}^n x_i$ 를 사용한다.
}

\defn{$\ex(\hat{\theta}) = \theta$ 를 만족하는 추정량 $\hat{\theta}$를 \textbf{불편추정량}(unbiased estimator)이라 한다.\\
추정량 $\hat{\mu} = \overline{X}$ 의 경우 $\ex(\overline{X}) = \mu$ 이므로 불편추정량이다.\footnote{표본분산 $\widehat{\sigma^2} = S^2$ 도 불편추정량이다. 사실 불편추정량이 되도록 $n-1$ 로 나눈 것이다...}
}

\defn{$\theta$의 추정량 $\hat{\theta}$의 표준편차를 \textbf{표준오차}(standard error)라 한다. 즉, $\mr{SE}(\hat{\theta}) = \sqrt{\var(\hat{\theta})}$ 이다. 표준오차는 추정량 $\hat{\theta}$의 흩어짐의 정도를 나타낸다.}\\

\defn{랜덤표본 $X_1, \dots, X_n$ 으로부터 얻어진 두 추정량 $L(X_1,\dots, X_n), U(X_1,\dots, X_n)$에 대하여 $$\pr(L(X_1,\dots, X_n)<\theta <U(X_1,\dots, X_n)) = 1-\alpha$$
가 성립할 때, 구간 $\big(L(X_1,\dots, X_n), U(X_1,\dots, X_n)\big)$를 $\theta$에 대한 $100(1-\alpha)\%$ \textbf{구간추정량}(interval estimator) 또는 \textbf{신뢰구간}(confidence interval)이라 한다. 주로
$$\pr(\theta \leq L(X_1,\dots, X_n)) = \pr(\theta \geq U(X_1,\dots, X_n)) = \alpha/2$$ 를 만족하는 $L(X_1,\dots, X_n), U(X_1,\dots, X_n)$ 를 사용한다.}

\thm{$X_1\dots, X_n \sim_{i.i.d} \mc{N}(\mu, \sigma^2)$ 일 때, 모평균 $\mu$ 에 대한 $100(1-\alpha)\%$ 신뢰구간은 $$\left(\overline{X}-z_{\alpha/2}\cdot \frac{\sigma}{\sqrt{n}}, \:\overline{X}+z_{\alpha/2}\cdot \frac{\sigma}{\sqrt{n}} \right)$$\\
\textbf{증명}. $\overline{X} \sim \mc{N}(\mu, \sigma^2/n)$ 이므로 $Z = \dfrac{\overline{X}-\mu}{\sigma/\sqrt{n}} \sim \mc{N}(0, 1)$ 이고, $$\pr(Z> z_{\alpha/2}) = \pr(Z<-z_{\alpha/2} )= \alpha/2$$ 이므로
$$\pr\left(\left|\frac{\overline{X}-\mu}{\sigma/\sqrt{n}}\right| \leq z_{\alpha/2}\right) = \pr\left(\overline{X}-z_{\alpha/2}\cdot \frac{\sigma}{\sqrt{n}}\leq \mu\leq \:\overline{X}+z_{\alpha/2}\cdot \frac{\sigma}{\sqrt{n}} \right)=1-\alpha$$\qed
}

\defn{구간추정량의 관측값 $L(x_1, \dots, x_n), U(x_1, \dots, x_n)$ 을 \textbf{구간추정값}(interval estimate) 또는 \textbf{신뢰구간}이라 한다.}\\
\textbf{예제}. 정규모집단에서의 모평균 $\mu$에 대한 $100(1-\alpha)\%$ 신뢰구간은 $$\left(\overline{x}-z_{\alpha/2}\cdot \frac{\sigma}{\sqrt{n}}, \:\overline{x}+z_{\alpha/2}\cdot \frac{\sigma}{\sqrt{n}} \right)$$\\
\textbf{참고}. $z_{0.05} = 1.645$, $z_{0.025} = 1.96$, $z_{0.005} = 2.576$ 임을 알아두면 좋다.\\

\prob{미국임상영양학회지에 실린 한 기록에 의하면, 중앙아메리카의 원주민을 대상으로 49명의 표본조사를 한 결과 혈청 내의 콜레스테롤 양이 157mg/L 였다고 한다. 이들 원주민의 혈청 내 콜레스테롤 양이 표준편차가 30인 정규분포를 따를 때, 모평균 $\mu$에 대한 95\% 신뢰구간을 구하여라.}

\thm{\textbf{모평균 $\mu$에 대한 추정 \underline{($\sigma$를 알 때)}}
\begin{itemize}
	\item 가정: $X_1, \dots, X_n \sim_{i.i.d} \mc{N}(\mu, \sigma^2)$, 모표준편차 $\sigma$는 알려진 값
	\item 추정량과 추정값: $\hat{\mu}(X_1, \dots, X_n) = \overline{X}$, $\hat{\mu}(x_1, \dots, x_n) = \overline{x}$
	\item 표준오차: $\mr{SE}(\hat{\mu}) = \sqrt{\var(\overline{X})} =\dfrac{\sigma}{\sqrt{n}}$
	\item $100(1-\alpha)\%$ 오차한계: $z_{\alpha/2}\cdot \dfrac{\sigma}{\sqrt{n}}$
	\item $100(1-\alpha)\%$ 신뢰구간: $\ds \left(\overline{X}-z_{\alpha/2}\cdot \frac{\sigma}{\sqrt{n}}, \:\overline{X}+z_{\alpha/2}\cdot \frac{\sigma}{\sqrt{n}} \right)$
	\item $100(1-\alpha)\%$ 신뢰구간의 길이: $2\cdot z_{\alpha/2}\cdot \dfrac{\sigma}{\sqrt{n}}$
\end{itemize}
모집단이 정규분포가 아닌 경우에는 $n$이 충분히 클 때 근사적으로 성립한다.
}

\prob{표준편차가 5인 모집단의 평균을 신뢰도 99\%로 추정할 때, 모평균 $\mu$와 표본평균 $\overline{X}$의 차이가 0.5 이하가 되도록 하려면 적어도 몇 개의 표본을 조사해야 하는가?}

\thm{$100(1-\alpha)$\% 오차한계를 $d$ 이하로 또는 $100(1-\alpha)$\% 신뢰구간의 길이를 $2d$ 이하로 하기 위한 최소 표본의 크기는 $n\geq \ds \left(\frac{z_{\alpha/2}\cdot \sigma}{d}\right)^2$ 인 최소의 정수이다.\\
\textbf{증명}. 연습문제로 남긴다. \qed}
\\
이제 모표준편차 $\sigma$를 모를 때 모평균 $\mu$를 추정하는 방법을 살펴보자. 이 경우에는 $\sigma$를 그 추정량인 표본표준편차 $S$로 대신해야 한다. 이렇게 표본표준편차를 이용해 표준화하는 과정을 \textbf{스튜던트화}(Studentize)라고 한다. 9장에서 $t$-분포를 배우면서 표본평균을 표본표준편차를 이용해 표준화하면 자유도가 $n-1$인 $t$-분포를 따름을 배웠다.\\

\thm{\textbf{모평균 $\mu$에 대한 추정 \underline{($\sigma$를 모를 때)},} 표본표준편차 $S$
	\begin{itemize}
		\item 가정: $X_1, \dots, X_n \sim_{i.i.d} \mc{N}(\mu, \sigma^2)$, 모표준편차 $\sigma$는 미지의 값
		\item 추정량과 추정값: $\hat{\mu}(X_1, \dots, X_n) = \overline{X}$, $\hat{\mu}(x_1, \dots, x_n) = \overline{x}$
		\item $100(1-\alpha)\%$ 오차한계: $t_{\alpha/2}(n-1)\cdot \dfrac{S}{\sqrt{n}}$
		\item $100(1-\alpha)\%$ 신뢰구간: $\ds \left(\overline{X}-t_{\alpha/2}(n-1)\cdot \frac{S}{\sqrt{n}}, \:\overline{X}+t_{\alpha/2}(n-1)\cdot \frac{S}{\sqrt{n}} \right)$
		\item $100(1-\alpha)\%$ 신뢰구간의 길이: $2\cdot t_{\alpha/2}(n-1)\cdot \dfrac{S}{\sqrt{n}}$
	\end{itemize}
	모집단이 정규분포가 아닌 경우에는 $n$이 충분히 클 때 근사적으로 성립한다.\\\\
\textbf{증명}. 신뢰구간만 증명한다.\\
$\overline{X} \sim \mc{N}(\mu, \sigma^2/n)$ 이므로 $T=\dfrac{\overline{X}-\mu}{S/\sqrt{n}} \sim t(n-1)$ 이고, $$\pr(T> t_{\alpha/2}(n-1)) = \pr(T<-t_{\alpha/2}(n-1) )= \alpha/2$$ 이므로
$$\begin{aligned}
1-\alpha &=\pr\left(\left|\frac{\overline{X}-\mu}{S/\sqrt{n}}\right| \leq t_{\alpha/2}(n-1)\right) \\&= \pr\left(\overline{X}-t_{\alpha/2}(n-1)\cdot \frac{S}{\sqrt{n}}\leq \mu\leq \:\overline{X}+t_{\alpha/2}(n-1)\cdot \frac{S}{\sqrt{n}} \right)
\end{aligned}
$$\qed
}
\\
\textbf{주의}. 위 방법들은 \underline{모집단이 정규분포를 따를 때} 사용할 수 있다. 따라서 표본으로 자료가 주어진 경우에는 \textbf{정규분포 분위수 대조도}(normal distribution quantile-quantile plot)을 통해 자료분포의 분위수와 정규분포의 분위수를 비교하여 모집단의 분포가 정규분포라는 가정을 검토해야 한다.\\\\
\textbf{참고}. $t$-분포표에서 자유도가 큰 경우에는 모든 자유도에 대해 확률 값이 계산되어 있지 않다. 이 경우에는 표의 자유도 중 실제 자유도보다 더 작으면서 가장 가까운 값을 사용한다.\\

\prob{정규분포를 따르는 모집단으로부터 64개의 자료를 관측한 결과 표본평균이 27, 표본표준편차가 5 였다. 모평균 $\mu$에 대한 99\% 신뢰구간을 구하여라.}
\\
통계적 추론을 할 때는 \textbf{통계적 가설}(statistical hypothesis)을 세워 모수에 대해 예상하거나 추측을 하고, 이를 \textbf{검정}한다.\\

\defn{자료로부터의 강력한 증거에 의해 입증하고자 하는 가설을 \textbf{대립가설}(alternative hypothesis)이라 하고 $H_1$ 으로 나타낸다. 그리고 반대 증거를 찾기 위해 상정된 가설을 \textbf{귀무가설}(null hypothesis)이라 하고, $H_0$ 으로 나타내며, 대립가설에 반대되는 가설이다. 두 가설 중 어느 가설을 택할지 통계적으로 결정하는 과정을 \textbf{가설 검정}(hypothesis test)이라 한다.}\\
\textbf{예제}. 어느 전구 공장의 기존 공정에서 전구의 수명이 평균 1200분, 표준편차 100분인 것으로 알려져 있다. 새로운 공정에 대하여 25개의 표본을 조사한 결과 표본평균이 1240분이었다. 새로운 공정으로 생산한 전구의 평균 수명 $\mu$에 대해 조사하려 한다.
 \begin{itemize}
	\item[(1)] 새로운 공법과 기존의 공법의 평균 수명이 다르다고 할 수 있는가?\\ 조사의 목적이 평균 수명이 달라졌다는 증거가 있는지 알아보기 위함이므로, \vspace{-3mm}
	\begin{center}
		$H_0$: $\mu=1200$ (평균 수명이 같다)\qquad $H_1$: $\mu \neq 1200$ (평균 수명이 다르다)
	\end{center}\vspace{-3mm}
	\item[(2)] 새로운 공법이 전구의 평균수명을 증가시켰다고 할 수 있는가?\vspace{-3mm}
	\begin{center}
		$H_0$: $\mu=1200$ (평균 수명이 같다)\qquad $H_1$: $\mu > 1200$ (평균 수명이 증가했다)
	\end{center}\vspace{-3mm}
\end{itemize}~

\defn{비교하는 값의 양쪽을 뜻하는 가설을 \textbf{양측가설}(two-sided hypothesis)이라 하고, 한쪽을 뜻하는 가설을 \textbf{단측가설}(one-sided hypothesis)이라 한다.}

\defn{귀무가설에 대한 반증의 강도를 제공하는 과정을 \textbf{유의성검증}(test of significance)이라 한다. 이 과정에서 사용되는 통계량을 \textbf{검정통계량}(test statistic)이라 한다.}\\
귀무가설과 대립가설은 모수에 관한 가설로 주어지므로, 유의성검증에서 찾고자 하는 증거는 그 모수의 추정량을 이용하여 찾게 된다. 따라서, 귀무가설에 반대되며 대립가설을 지지하는 증거는 모수의 추정값이 귀무가설 $H_0$에서 주어지는 모수의 값으로부터 대립가설 $H_1$의 방향으로 멀리 떨어질수록 강해진다.\\\\
위 예제의 (2)에서 귀무가설과 대립가설이 각각 모평균에 관한 가설이므로, 표본평균 $\overline{X}$를 이용하여 유의성검증에서의 증거를 찾게 된다. 그렇다면 가설에 대한 반증의 강도는 어떻게 측정할지 의문이 들기 마련이다. 다음과 고려해 보자.
\textbf{귀무가설 $H_0$가 사실일 때, 실제 관측값보다 더욱 $H_0$에 반대되며, $H_1$을 지지하는 조사 결과를 얻을 확률은 얼마인가?}\\\\
위 예제에서 실제 관측값보다 더욱 $H_0$에 반대되며 $H_1$을 지지하는 조사의 결과는 $\overline{X}\geq 1240$ 이다. 이에 대한 확률을 계산해 보면
$$\begin{aligned}
\pr\left(\overline{X}\geq 1240\,|\, H_0 \text{ 가 참}\right) &= \pr\left(\frac{\overline{X}-1200}{100/\sqrt{25}}\geq \frac{1240-1200}{100/\sqrt{25}} \,\biggr|\, \mu=1200\right)\\
&=\pr(Z\geq 2) = 0.0228
\end{aligned}$$
이다. 따라서, 무한히 같은 조사를 해도, 실제 조사 결과 $\overline{x}=1240$ 보다 더욱 강한 $H_0$의 반증을 얻을 기회는 2.28\% 이므로, 위 확률은 이 관측값이 얼마나 일어나기 어려운 것인지 나타내 준다. 즉, 귀무가설 $H_0$가 사실이 아님을 강하게 시사한다고 할 수 있다.\\

\defn{검정통계량이 실제 관측된 값보다 대립가설을 지지하는 방향으로 더욱 치우칠 확률로서 귀무가설 $H_0$하에서 계산된 값을 \textbf{유의확률}(significance probability) 또는 $P$-값($p$-value)라 한다. 유의확률이 작을수록 $H_0$에 대한 반증이 강한 것을 뜻하며, 유의확률을 계산할 때는 $H_0$ 하에서 검정통계량의 분포를 이용한다.}\\
반증이 ``강하다"는 상대적인 표현이므로, 가설검정을 할 때, 미리 기준값을 정해두고 유의확률을 그 기준값과 비교한다.

\defn{귀무가설 $H_0$에 대한 반증의 강도에 대하여 미리 정해둔 기준값을 \textbf{유의수준}(significance level)이라 부르고, 흔히 $\alpha$로 나타낸다.}\\
유의수준으로는 주로 $\alpha=0.1$, 0.05, 0.01 을 사용하며, 유의수준이 $\alpha$인 것은 [귀무가설에 대한 반증이 조사 결과보다 강하게 나타날 확률]이 $\alpha$ 이하일 것을 요구하는 것이다. 유의확률이 지정된 유의수준 $\alpha$ 이하로 나타나면, \textbf{유의수준 $\alpha$ 에서 유의하다}(statistically significant)고 하며, 이는 귀무가설에 대한 반증의 강도가 지정된 수준보다 강함을 의미한다. 따라서 귀무가설을 \textbf{기각}(reject)하고 대립가설을 채택한다.\\

\defn{가설검정에서 오류의 종류
\begin{center}
	\begin{tabular}{c|c|c}
		검정결과\textbackslash 실제상황 & $H_0$가 참 & $H_1$이 참\\
		\hline
		$H_0$ 채택 & 옳은 결정 & \small \textbf{제 2종의 오류}(Type II Error $\beta$) \\
		\hline
		$H_0$ 기각 & \small \textbf{제 1종의 오류}(Type I Error $\alpha$) & 옳은 결정
	\end{tabular}
\end{center}
}\\
유의수준을 $\alpha$로 지정한다는 것은 제 1종의 오류를 범할 확률의 허용한계를 $\alpha$로 미리 정해주는 것이다.\\

\defn{정해진 유의수준에 따라, [귀무가설을 기각하게 되는 검정통계량의 관측값]의 범위를 \textbf{기각역}(rejection region)이라 한다.\footnote{기각역과 기각역이 아닌 영역을 나누는 기준이 되는 경계값을 critical value 라고도 한다.}}\\
\textbf{가설검정의 절차}
\begin{itemize}
	\item 귀무가설과 대립가설을 설정한다. 유의성검증은 \textbf{귀무가설에 대한 반증의 강도}를 알아보기 위함임을 고려한다.
	\item 유의수준 $\alpha$를 지정한다. 귀무가설에 대한 반증의 강도가 어느 정도이어야 하는지 고려한다.
	\item 자료로부터 검정통계량의 관측된 값을 계산한다. 검정통계량은 가설에 관계되는 모수의 추정량을 이용한다.
	\item (\textbf{유의확률} 사용) 검정통계량의 관측값으로부터 유의확률을 계산하여 유의수준보다 작으면 $H_0$를 기각한다. 그렇지 않으면 $H_0$를 채택한다.
	\item (\textbf{기각역} 사용) 유의수준에 대한 기각역을 찾아 검정통계량의 관측값이 기각역에 속하면 $H_0$를 기각한다. 그렇지 않으면 $H_0$를 채택한다.
\end{itemize}~

\thm{\textbf{모평균 $\mu$에 대한 유의성검증 \underline{($\sigma$를 알 때)}} - $Z$ 검정
\begin{itemize}
	\item 가정: $X_1, \dots, X_n \sim_{i.i.d} \mc{N}(\mu, \sigma^2)$, 모표준편차 $\sigma$는 알려진 값. 혹은 모집단이 정규분포는 아니지만 $n$이 충분히 큰 경우 (중심극한정리)
	\item 귀무가설 $H_0$: $\mu = \mu_0$
	\item \textbf{검정통계량}: $\ds Z = \frac{\overline{X} - \mu_0}{\sigma/\sqrt{n}} \sim_{H_0} \mc{N}(0, 1)$,\quad \textbf{관측값}: $\ds z = \frac{\overline{x}-\mu_0}{\sigma/\sqrt{n}}$
	\item 대립가설의 형태에 따른 기각역과 유의확률
	\begin{center}
		\begin{tabular}{c|c|c}
			대립가설 & 유의수준 $\alpha$에서의 기각역 & 유의확률\\ \hline
			$H_1$: $\mu >\mu_0$ & $z\geq z_\alpha$ & $\pr\left(Z\geq z\right)$\\
			$H_1$: $\mu <\mu_0$ & $z\leq - z_\alpha$ & $\pr\left(Z\leq z\right)$\\
			$H_1$: $\mu \neq \mu_0$ & $\left|z\right|\geq z_{\alpha/2}$ & $\pr\left(\left|Z\right|\geq \left|z\right|\right)$
		\end{tabular}
	\end{center}
\end{itemize}
}

\prob{한 제약 회사의 연구개발부에서 특정 약품의 주성분의 농도가 평균이 $\mu$이고 표준편차가 0.0123 인 정규분포를 따른다고 한다. 이 연구개발부에서는 이론적으로 주성분의 농도가 0.85 일 것이라고 예상하는 제조법을 창안하였다. 제조와 측정의 기술적, 비용적 측면을 고려하여 3번 반복 측정한 결과 주성분 농도의 평균이 $\overline{x}=0.8404$ 이었다. 이 제조법에 의한 주성분의 실제 농도 $\mu$가 0.85 가 아니라고 의심할 만한 증거가 있는가? 유의수준 0.05에서 검정하여라.}\\

\thm{\textbf{모평균 $\mu$에 대한 유의성검증 \underline{($\sigma$를 모를 때)}} - $t$ 검정
	\begin{itemize}
		\item 가정: $X_1, \dots, X_n \sim_{i.i.d} \mc{N}(\mu, \sigma^2)$, 모표준편차 $\sigma$는 미지의 값. 혹은 모집단이 정규분포는 아니지만 $n$이 충분히 큰 경우 (중심극한정리)
		\item 귀무가설 $H_0$: $\mu = \mu_0$
		\item \textbf{검정통계량}: $\ds T = \frac{\overline{X} - \mu_0}{S/\sqrt{n}}\sim_{H_0}t(n-1)$,\quad \textbf{관측값}: $\ds t = \frac{\overline{x}-\mu_0}{s/\sqrt{n}}$ ($s$: 표본표준편차)
		\item 대립가설의 형태에 따른 기각역과 유의확률
		\begin{center}
			\begin{tabular}{c|c|c}
				대립가설 & 유의수준 $\alpha$에서의 기각역 & 유의확률\\ \hline
				$H_1$: $\mu >\mu_0$ & $t\geq t_{\alpha}(n-1)$ & $\pr\left(T\geq t\right)$\\
				$H_1$: $\mu <\mu_0$ & $t\leq -t_{\alpha}(n-1)$ & $\pr\left(T\leq t\right)$\\
				$H_1$: $\mu \neq \mu_0$ & $\left|t\right|\geq t_{\alpha/2}(n-1)$ & $\pr\left(\left|T\right|\geq \left|t\right|\right)$
			\end{tabular}
		\end{center}
	\end{itemize}
}

\prob{전구를 생산하는 회사에서 현재 생산하는 전구의 수명은 평균이 1950시간인 정규분포를 따른다고 알려져 있다. 새롭게 개발중인 전구의 평균수명 $\mu$가 기존의 전구보다 수명이 더 길다고 할 수 있는지 확인하기 위해 9개의 시제품을 생산하여 그 수명시간을 조사한 결과가 다음과 같다.
\begin{center}
	2000, 1975, 1900, 2000, 1950, 1850, 1950, 2100, 1975
\end{center} 적절한 가설을 세우고, 유의수준 5\%에서 검정하여라.
}\\\\

\defn{제 2종의 오류를 범하지 않을 확률을 \textbf{검정력}(power)이라 한다. 제 2종의 오류를 범할 확률을 주로 $\beta$로 두며, 이때 검정력은 $1-\beta$가 된다.}

\prob{연습문제 10.22 에서 실제로 $\mu=0.84$ 일 때, 검정력을 구하여라.}\\\\

\thm{표준편차를 알고있는 정규모집단 $\mc{N}(\mu, \sigma)$ 에서의 표본에 대해, 유의수준 $\alpha$인 $H_0$: $\mu = \mu_0$, $H_1$: $\mu>\mu_0$ 의 검정에서 실제로 $\mu=\mu_1$($>\mu_0$) 일 때, 제 2종의 오류를 범할 확률이 $\beta$ 이하가 되기 위한 표본의 크기 $n$은 다음을 만족하는 최소의 정수이다.
$$n \geq \left(\frac{z_\alpha+z_\beta}{(\mu_1-\mu_0)/\sigma}\right)^2$$
\textbf{증명}. 제 2종의 오류를 범할 확률이 $\beta$ 이하가 되어야 하므로,
$$\begin{aligned}
	\pr\left(\frac{\overline{X}-\mu_0}{\sigma/\sqrt{n}} < z_\alpha \,\biggr|\, \mu=\mu_1 \right) &= \pr\left(\frac{\overline{X}-\mu_1}{\sigma/\sqrt{n}} + \frac{\mu_1-\mu_0}{\sigma/\sqrt{n}} < z_\alpha \,\biggr|\, \mu=\mu_1 \right)\\ &=\pr\left(\frac{\overline{X}-\mu_1}{\sigma/\sqrt{n}} < z_\alpha -\frac{\mu_1-\mu_0}{\sigma/\sqrt{n}} \,\biggr|\, \mu=\mu_1\right) < \beta
\end{aligned}$$ 가 성립해야 한다. $\ds \frac{\overline{X}-\mu_1}{\sigma/\sqrt{n}}\sim \mc{N}(0, 1)$ 이므로,
$$-z_\beta \geq z_\alpha -\frac{\mu_1-\mu_0}{\sigma/\sqrt{n}} $$ 이 성립하고, 이를 $n$에 대해 정리하면 원하는 결론을 얻는다. \qed
}

\prob{한 제약회사에서 생산하고 있는 기존의 진통제는 진통 효과가 나타나는 시간이 평균 30분, 표준편차 5분인 것으로 알려져 있다. 회사의 연구소에서 진통 효과가 더 빨리 나타날 것으로 기대되는 새로운 진통제를 개발하였다. 새로운 진통제의 진통효과가 더 빠른가를 확인하기 위하여, 50명의 환자를 랜덤추출하여 새로운 진통제에 의해 그 효과가 나타나는 시간을 측정한 결과 평균이 28.5분이었다. 새로운 진통제에 의한 진통 효과가 나타나는 시간이 표준편차 5분인 정규분포를 따른다고 하자. 다음 물음에 답하여라.
\begin{enumerate}
	\item[(1)] 적절한 가설을 유의수준 5\%에서 검정하여라.
	\item[(2)] 유의수준 5\%에서 검정할 때, 제 1종의 오류를 범할 확률을 구하여라.
	\item[(3)] 실제로 $\mu=28$ 일 때, 제 2종의 오류를 범할 확률이 $\beta=0.10$ 이하가 되도록 하며, 유의수준 5\%인 검정법을 사용하려고 한다. 이 때, 요구되는 표본의 최소 크기를 구하여라.
\end{enumerate}
}

\pagebreak

\input{./sec/11.tex}
\input{./sec/12.tex}

\end{document}
